  % Created 2018-06-21 Thu 12:30
\documentclass[8pt]{beamer}
\usepackage[sc,osf]{mathpazo}   % With old-style figures and real smallcaps.
\linespread{1.025}              % Palatino leads a little more leading
% Euler for math and numbers
\usepackage[euler-digits,small]{eulervm}
%\documentclass[10pt]{llncs}
%\usepackage{llncsdoc}
\usepackage{minted}
\usepackage[utf8]{inputenc}
\usepackage[T1]{fontenc}
\usepackage{fixltx2e}
\usepackage{graphicx}
\usepackage{longtable}
\usepackage{float}
\usepackage{tikz}
\usepackage{tikz-cd}
\usepackage{wrapfig}
\usepackage{rotating}
\usepackage{changepage}
\usepackage[normalem]{ulem}
\usepackage{amsmath}
\usepackage{textcomp}
\usepackage{marvosym}
\usepackage{wasysym}
\usepackage{amssymb}
\usepackage{hyperref}
\usepackage{polynom}
\renewcommand{\mod}[1]{\left( \texttt{mod}~#1 \right)}
\newcommand{\N}{\mathbb N}
\newcommand{\Z}{\mathbb Z}
\newcommand{\Q}{\mathbb Q}
\newcommand{\C}{\mathbb C}
\newcommand{\cat}[1]{\mathsf{#1}}
\newcommand{\cSet}{\cat{Set}}
\newcommand{\nt}{\Rightarrow}
\tolerance=1000
\usetheme{Antibes}
\author{Siddharth Bhat}
\date{October 23th, 2021}
\institute{IIIT Hyderabad}
\title{Mathematical structures for word embeddings}
\hypersetup{
  pdfkeywords={},
  pdfsubject={},
  pdfcreator={Emacs 24.5.1 (Org mode 8.2.10)}}
\begin{document}

\maketitle

\begin{frame}[label=sec-1]{What is a word embedding?}
  \pause
\begin{itemize}
  \item Map words to \emph{mathematical objects}.  \pause
  \item Semantic ideas on words $\simeq$ mathematical operations on these objects. \pause
  \item Most common: \emph{vector embeddings} (\texttt{word2vec}) \pause
\end{itemize}
\end{frame}

\begin{frame}{Is \texttt{word2vec} sensible?}

\end{frame}

\begin{frame}{Part I: What's a philosopher to do?}

\end{frame}

\begin{frame}{Are sets hiding in \texttt{word2vec}?}

\end{frame}

\begin{frame}{What does this buy us anyway?}

\end{frame}

\begin{frame}{Take-aways}

\end{frame}

\begin{frame}{Pat II: What's a geometer to do?}

\end{frame}

\begin{frame}{From vectors to subspaces}

\end{frame}

\begin{frame}{A research agenda, and carrying the baton forward}
\end{frame}

\begin{frame}{Conclusion}
  \begin{itemize}
    \item \texttt{word2vec} is performant but poorly understood.
    \item We extract fuzzy set embeddings from \texttt{word2vec}, appeasing Montague!
    \item We ponder on the geometry of \texttt{word2vec}, and indicate potential extensions.
    \item TL;DR: Mathematical modelling (fuzzy sets, grassmanians) is useful to extend empirical results (\texttt{word2vec})!
  \end{itemize}
\end{frame}


\end{document}

