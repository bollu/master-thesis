\documentclass{article}
\begin{document}
With the growing use of natural language processing tools (NLP) in wide-ranging
applications, it becomes imperative to understand why NLP models are as
successful as they are.  In particular, it is essential to understand what
mathematical structures precisely underpin these models. To investigate the
mathematical models of NLP knowledge representations, we focus on the setting
of unsupervised word embeddings, due to the presence of robust models, and a
seemingly simple mathematical structure. We find that even in this restricted
setting, there are subtle, cross-cutting concerns as to how the model learns —
beginning from the high level description of learning a “vector”, to the low
level details of implementa- tions of these algorithms, including
initialization and gradient calculation. We build a theory of knowledge
representations for word embeddings, inspired by two seemingly unrelated ideas
(a) Montague semantics [24], a linguistic theory of meaning which ascribes
set-theoretic meaning to language, and (b) abstract interpretation, a
mathematical theory of creating com- putable approximations to uncomputable
semantics of programs. We find that syntheszing these ideas provide a way to
extract fuzzy-set embeddings from existing word-embeddings, which provides the
full range of set-theoretic operations (union, intersection, and others), along
with probabilistic operations (KL divergence, entropy) which are used to
perform pol- ysemy detection and word sense disambiguation, while retaining
performance on regular word embeddings tasks such as similarity. Next, we turn
our attention to generalizing the word embedding training regime by extracting
the geometry which is currently held implicit.  This leads us to formulate a
general theory of learning word representations on Lie groups.  In summary, we
provide insight into the mathematical structure of word representations.
\end{document}
